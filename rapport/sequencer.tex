\documentclass[a4paper, 12pt, titlepage]{article}
\usepackage[utf8]{inputenc}
\usepackage[left=2cm,right=2cm,top=2cm,bottom=2cm]{geometry}
\usepackage[pdftex]{graphicx}
\usepackage[final]{pdfpages}
\usepackage[frenchb]{babel}%Pas important
\newcommand{\arc}[2]{$(#1~\rightarrow~#2)$}

\begin{document}
\newcommand{\myTitleOfTitlepage}{Séquenceur ADN}
\newcommand{\courseTitleOfTitlepage}{Algorithmique et bioinformatique}

\begin{titlepage}
\begin{center}
\includegraphics[scale=1.50]{figures/UMONS.jpg}\\[0.4cm]
\includegraphics[scale=0.30]{figures/FS_Logo.jpg}\\[3cm]
{\huge \courseTitleOfTitlepage}\\
\vspace{0.9mm}
\rule{9.5cm}{0.5mm}\\[0.5cm]
{\huge \bfseries \myTitleOfTitlepage}\\[0.2cm]
\rule{9.5cm}{0.5mm}\\[7cm]
      \begin{flushleft} \large
        Guillaume \textsc{Huysmans} 140223\\
        Jason \textsc{Bury} 130538\\
        Master 1 en Sciences informatiques\\
      \end{flushleft}
   \vfill
  {\large \today}
\end{center}
\end{titlepage}
\section{Introduction}
\subsection{Énoncé}
Le projet consiste à l'implémentation d'un séquenceur d'ADN,
c'est-à-dire reconstruire la séquence ADN cible à partir d'un ensemble de fragments.
Pour ce faire, une heuristique de type \emph{greedy} devra être utilisée pour
reconstruire un chemin hamiltonien (ce problème est NP-complet) dans un graphe
où les arcs sont des fragments et
les arcs sont pondérés avec les scores d'alignements semi-globaux.
Nous devons prendre en compte que certains fragments sont inclus à d'autre et
qu'il faudra parfois prendre le complémentaires inversé d'un fragment de
l'ensemble de départ.

\section{Consensus}

La fin du premier noeud est ajoutée à une file de positions à traiter.
Ensuite, une liste d'arcs est parcourue,
le début et la position juste avant la fin de l'alignement (auquel des gaps ont
pu être ajoutées selon les alignements précédents) sont ajoutées
à la file de priorité. Dès qu'un alignement est décalé par rapport à celui qui
le précède (ceci évite des problèmes liés aux fragments inclus),
une position minimale est extraite de la file et indique jusqu'où
dont être fait le vote. Les pointeurs dans les alignements sont incrémentés et
dès lors qu'ils atteignent la taille de ceux-ci, l'alignement est retiré de
la liste des alignements à traiter.

Dans le pire des cas, on pourrait imaginer une suite de fragments préfixes les
uns des autres, triés par ordre croissant de longueur. Le nombre de bases lues
serait en $O(n^2)$ avec $n$ la longueur du dernier fragment,
ce qui reste polynomial.

En pratique, ils seront bien moins nombreux à être examinés en même temps.


\section{Les complémentaires inversés dans l'algorithme Greedy}%explique comment on gère la suppressions des complémentaires inversés
Au cours de l'exécution de l'algorithme glouton, lorsqu'un arc est ajouté au chemin hamiltonien, les complémentaires inversés de ses noeuds ne doivent plus être pris en compte.
Pour cela, nous avons modifié l'algorithme glouton et ajouté une fonction dans la structure union-find:
en ajoutant un arc \arc{f}{g} au chemin hamiltonien, nous mettons les booléens \texttt{in} et \texttt{out} de leur complémentaire inversé à vrai.
Ainsi les arcs issus de ces noeuds ne seront jamais ajoutés dans le chemin.
De plus, nous avons ajouté une méthode dans l'objet \texttt{UnionFind} afin de décrémenter le compteur d'ensembles distincts dans la structure.
Cette méthode sera appelée deux fois : chaque fois qu'un arc \arc{f}{g} est ajouté au chemin.
En effet, si l'arc \arc{f}{g} est ajouté au chemin, nous savons que la structure \texttt{UnionFind} contient les ensembles \{$\bar{f}$\} et \{$\bar{g}$\} puisqu'ils n'ont jamais fusionné avec un autre ensemble.
Puisqu'ils ne seront plus jamais utilisés, il n'est pas nécessaire de
réellement les supprimer, il suffit de décrémenter de deux le nombre d'ensembles distincts.
Ce compteur d'ensembles distincts est utilisé pour arrêter l'algorithme
lorsqu'il n'y a plus qu'un ensemble dans l'\texttt{UnionFind}.

\section{Les inclusions de fragments}
Dans le calcul du consensus, la première chose qui est faite est de calculer
l'ordre de traitement des fragments. L'alignement de $s$ et $t$ calcule aussi
$\delta$, l'écart entre le premier nucléotide de $t$ et celui de $s$.
$t$ commence avant $s$ (et $t \subseteq s$) si et seulement si $\delta<0$.

Nous aurions pu nous simplifier la tâche et supprimer les fragments inclus à
d'autres mais leurs voix auraient été échangées contre un calcul plus simple
du consensus.
%TODO exemple simple ?

\section{Optimisations}
Dans cette section, nous expliquerons les techniques que nous avons utilisé dans le but d'améliorer les performances de notre programme, en temps de calcul ou en mémoire.

\subsection{Classe complementary}
Puisque certains fragments peuvent être le complémentaire inversé d'un fragment de la cible, alors nous devons créer et ajouter les complémentaires inversés de tous les fragments dans le graphe avant l'application de l'algorithme Greedy.
Au lieu de créer ces fragments, nous associons à chaque Sequence $f$, un objet $\bar{f}$ qui hérite de Sequence et dont la fonction \texttt{get($i$)} retournera le $i$\textsuperscript{ème} symbole à partir de la fin de $f$ mais complémentarisé.
Ceci afin d'éviter d'occuper deux fois la mémoire nécessaire pour les fragments chargés à partir du fichier.

\subsection{Identifiant des fragments}
Pour améliorer les performances de l'algorithme Greedy, un identifiant est donné à chaque fragment de cette manière:\\
Soit $n$ le nombre de fragments chargés depuis le fichier. Soit $0 \le i < 2n$.
\begin{itemize}
 \item \textbf{Si $i < n$}, Le fragment $i$ est le $(i+1)$\textsuperscript{ème} fragment chargé.
 \item \textbf{Si $i \ge n$}, Le fragment $i$ est le complémentaire inversé du fragment $i-n$.
\end{itemize}
Aucun identifiant n'est enregistré en mémoire. Ils indiquent la position dans la liste de fragments.
Et les noeuds du graphes sont maintenant représentés par un identifiant.

Cela améliore les performances. En effet,\\
Cela permet de remettre l'ensemble des arcs du chemin hamiltonien dans l'ordre.
Lors de l'algorithme Greedy, chaque fois qu'un arc \arc{f}{g} est ajouté au chemin, on enregistre dans un tabeau (indicé à partir de 0) en position $f-1$ l'indice de l'arc dans le tableau où on vient de l'ajouter.
Une fois l'algorithme Greedy terminé,\\
On trouve en temps linéaire le premier fragment du chemin (c'est le seul dont le \texttt{in} est à \texttt{faux}).\\
Puis en temps linéaire, on boucle en prenant chaque fois l'arc \arc{f}{g} suivant (on le trouve dans le tableau en position $f-1$) et le prochain arc se trouve dans le tableau en position $g-1$.\\
Et donc remettre les arcs dans l'ordre se fait en $O(n)$ où $n$ est le nombre de fragments.
Si les noeuds n'étaient pas des identifiants, il aurait fallu une recherche pour trouver l'arc suivant et donc on aurait eu un algorithme en $O\left(n^2\right)$.

De plus, lorsqu'un arc \arc{f}{g} est ajouté au chemin hamiltonien, trouver les complémentaires inversés de $f$ et $g$ pour mettre leur \texttt{in} et \texttt{out} à \texttt{vrai} se fait en $O(1)$.\\
Ensuite, il n'y a plus besoin de recherche dans la liste de fragment pour chaque appel à la méthode \texttt{find} dans l'\texttt{UnionFind} si les noeuds dans l'\texttt{UnionFind} sont représentés par un fragment.

Nous aurions pu nous passer d'identifiant en créant un objet \texttt{Noeud} contenant un fragment, une référence vers le noeud du complémentaire inversé, la position de l'arc suivant dans le chemin, le parent dans l'\texttt{UnionFind}, \texttt{in} et \texttt{out}.
Mais cela rend l'implémentation plus compliquée et beaucoup moins éléguante.

\subsection{Multithreading}
Avant de passer à l'algorithme Greedy, il faut créer tous les arcs et donc calculer les scores d'alignement entre chaque pair de fragments.
Tous les calculs de score peuvent se faire en parallèle. Deux méthodes de parallèlisations sont envisageables:
\begin{enumerate}
 \item Pour chaque thread on défini une part égale de l'ensemble des arcs à calculer.
 \item Chaque fois qu'un thread a calculé un arc, il s'en réserve un autre dans une mémoire partagée avec les autres threads.
\end{enumerate}
Nous avons choisi la premières méthodes car nous observons que les différences de temps de calcul entre chaque thread sont égales à 1 seconde près. %TODO refaire les tests et indiquer les résultats
Ce qui signifie que leur charge sont équilibrées.
Et de plus, beaucoup d'appels système auraient eu lieu si la deuxième méthodes était utilisée.

\subsection{Même score pour plusieurs arcs}
Les arcs utilisés sont orientés. Le poid d'un arc \arc{f}{g} est le score de l'alignement où soit on aligne le suffixe de $f$ au préfixe de $g$, soit $f$ est inclu à $g$.
Après avoir calculé la matrice d'alignement, ce score est le maximum des scores sur la dernière ligne.
Mais ce score est aussi égal au score de \arc{\bar{g}}{\bar{f}} car l'alignement est exactement le même.
Le maximum des scores de la dernière colonne de la matrice donne le score de \arc{g}{f} et donc aussi le score de \arc{\bar{f}}{\bar{g}}.
4 scores peuvent donc être calculés avec une seule matrice d'alignement.

\subsection{Type des données} %TODO modification faite chez jason mais pas encore pushée
La valeur d'un score peut être enregistré en 2 octets (\texttt{short} en Java).
Effectivement, on considère que les fragments sont tous de taille $\le 1000$. %TODO donner la taille maximale d'un fragment
Une borne inférieure au score est -2000 (que des gaps) %TODO pas oublier de changer si changement des constantes de score
et une borne supérieure est 1000 (alignement total).
Un nombre en 2 octets se trouve entre -32768 et 32767.
\end{document}
