\subsection{Consensus}
Les alignements deux à deux ne peuvent pas être utilisés tels quels :
d'une part, les gaps doivent parfois être reportés et
d'autre part, un vote doit être effectué lorsque des fragments se chevauchent.

\subsubsection{Report des gaps et ordonnancement}
Les fragments sont d'abord ordonnés par position croissante,
relative à celle du premier fragment.
Pour ce faire, le premier fragment est ajouté à la position 0 avant de
parcourir tous les arcs \arc{s}{t} du chemin hamiltonien :
les gaps de l'alignement précédent sont ajoutés à celui de $t$ avec $s$ et
l'écart entre les alignements de $s$ et de $t$ (négatif si $s\subseteq t$) est
ajouté à la position courante. Cette position et ce nouvel alignement de $t$
sont ajoutés à une file de priorité.

Le calcul de l'alignement prend toujours un temps en $O(n^2)$, avec $n$ la
longueur du plus grand fragment. Le traitement dans \texttt{getAlignment}
est en $O(n)$ (parcours de toute la diagonale, dans le pire des cas).
L'ajout de $n$ éléments à une file de priorité se fait en $n\times O(1)=O(n)$.
Au total, on obtient du $O(n^2)$.

\subsubsection{Construction du contig}
Ensuite, on prend l'alignement au sommet de la file de priorité
(s'il est répété, on supprime ses répétitions) et on l'ajoute à
la liste des alignements en cours d'utilisation.
Ensuite, on itère tant qu'on n'a pas atteint le prochain élément de la file
(ou, si elle est vide, traité toute la liste de fragments) :
un vote est fait, les pointeurs sont avancés et les alignements terminés (ceux
dont le pointeur a dépassé la fin) sont retirés de la liste.
On recommence le processus jusqu'à ce que la file soit vide.

Le vote se fait par majorité et en cas d'égalité, $gap < A < C < G < T$.

Dans le pire des cas, on pourrait imaginer une suite de fragments préfixes les
uns des autres (l'entrée serait un genre de grand escalier descendant),
triés par ordre croissant de longueur. Le nombre de bases lues
serait en $O(n^2)$ avec $n$ la longueur du dernier fragment,
ce qui reste polynomial : au premier vote, on regarde $n$ nucléotides, au
second, $n-1$... et par la formule de Gauss, on trouve une complexité en
$n\times(n+1)/2=O(n^2)$.

En pratique, ils seront bien moins nombreux à être examinés en même temps.

\subsubsection*{Les inclusions de fragments}
Pour rappel, si nous avons deux fragments $f$ et $g$ et que lors de l'alignement, $f$ est inclus à $g$,
alors on a un arc \arc{f}{g} car le sens de l'arc signifie juste que l'alignement se termine à la fin de $f$.
Dans le calcul du consensus, à la première étape, celle qui consiste à calculer l'ordre de traitement des fragments,
nous recherchons pour chaque alignement, $\Delta$, la position du début de l'alignement à partir de la position du premier caractère de $f$.
Et cette position peut donc être négative dans le cas où $f$ est inclus à $g$ et que l'alignement commence avant le premier caractère de $f$.

En obtenant le $\Delta$ pour chaque alignement, nous pouvons donc obtenir une file de priorité contenant, dans l'ordre, les positions absolues de débuts d'alignements.
Par position absolue, nous entendons la position dans le contig sans suppression de gap.
Cette file de priorité nous permettra donc de savoir à partir de quelle position le nombre d'alignement à prendre en compte pour le consensus d'un symbole change.


Nous aurions pu nous simplifier la tâche et supprimer les fragments inclus à
d'autres mais leurs voix auraient été échangées contre un calcul plus simple
du consensus.
%TODO exemple simple ?
