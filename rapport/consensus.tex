\section{Consensus}

La fin du premier noeud est ajoutée à une file de positions à traiter.
Ensuite, une liste d'arcs est parcourue,
le début et la position juste avant la fin de l'alignement (auquel des gaps ont
pu être ajoutées selon les alignements précédents) sont ajoutées
à la file de priorité. Dès qu'un alignement est décalé par rapport à celui qui
le précède (ceci évite des problèmes liés aux fragments inclus),
une position minimale est extraite de la file et indique jusqu'où
dont être fait le vote. Les pointeurs dans les alignements sont incrémentés et
dès lors qu'ils atteignent la taille de ceux-ci, l'alignement est retiré de
la liste des alignements à traiter.

Dans le pire des cas, on pourrait imaginer une suite de fragments préfixes les
uns des autres, triés par ordre croissant de longueur. Le nombre de bases lues
serait en $O(n^2)$ avec $n$ la longueur du dernier fragment,
ce qui reste polynomial.

En pratique, ils seront bien moins nombreux à être examinés en même temps.
