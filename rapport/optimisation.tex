\section{Optimisations}
Dans cette section, nous expliquerons les techniques que nous avons utilisées dans le but d'améliorer les performances de notre programme, en temps de calcul ou en mémoire.

\subsection{Classe Complementary}
Puisque certains fragments peuvent être le complémentaire inversé d'un fragment
de la cible, nous devons créer et ajouter les complémentaires inversés de
tous les fragments dans le graphe avant l'application de l'algorithme Greedy.
Au lieu de créer ces fragments, nous associons à chaque séquence $f$ un objet
$\bar{f}$ qui hérite de \texttt{Sequence} et dont la méthode \texttt{get($i$)}
retournera le complémentaire du $i$\textsuperscript{ème} symbole
à partir de la fin de $f$.
Cela évite d'occuper deux fois la mémoire nécessaire
pour les fragments chargés à partir du fichier.

\subsection{Identifiant des fragments}
Pour améliorer les performances de l'algorithme Greedy, chaque fragment est
numéroté. Soient $n$ le nombre de fragments chargés depuis le fichier et $0 \le i < 2n$.
\begin{itemize}
 \item Si $i < n$, le fragment $i$ est le $(i+1)$\textsuperscript{ème} fragment chargé.
 \item Si $i \ge n$, le fragment $i$ est le complémentaire inversé du fragment $i-n$.
\end{itemize}

Les identifiants sont utilisés dans les arcs pour indicer la liste des fragments.

Cela améliore les performances. En effet, cela permet de remettre l'ensemble
des arcs du chemin hamiltonien dans l'ordre.
Lors de l'algorithme Greedy, à chaque fois qu'un arc \arc{f}{g} est ajouté
au chemin, on enregistre dans un tableau (indicé à partir de 0)
en position $f-1$ l'indice de l'arc dans le tableau où on vient de l'ajouter.
Une fois l'algorithme Greedy terminé, on trouve en temps linéaire
le premier fragment du chemin (c'est le seul dont le flag \texttt{in} est faux).
Ensuite, toujours en temps linéaire, on boucle en prenant chaque fois
l'arc \arc{f}{g} suivant que l'on trouve en position $f-1$ et
le prochain arc se trouve en position $g-1$.
Ainsi, remettre les arcs dans l'ordre se fait en $O(n)$
où $n$ est le nombre de fragments.
Si les noeuds n'étaient pas des identifiants, il aurait fallu effectuer
une recherche pour trouver l'arc suivant et
on aurait donc eu un algorithme en $O\left(n^2\right)$.

De plus, lorsqu'un arc \arc{f}{g} est ajouté au chemin hamiltonien,
trouver les complémentaires inversés de $f$ et $g$ pour mettre
leur \texttt{in} et \texttt{out} à vrai se fait en $O(1)$.
Ensuite, il n'y a plus besoin d'effectuer des recherches dans
la liste de fragments pour chaque appel à la méthode \texttt{find}
dans l'union-find si les noeuds sont représentés par un fragment.

Nous aurions pu nous passer d'identifiant en créant un objet \texttt{Node}
contenant un fragment, une référence vers le noeud du complémentaire inversé,
la position de l'arc suivant dans le chemin,
le parent dans l'union-find, \texttt{in} et \texttt{out} mais
cela rend l'implémentation plus compliquée et beaucoup moins élégante.

\subsection{Multithreading}
Avant de passer à l'algorithme Greedy, il faut créer tous les arcs et donc
calculer les scores d'alignement entre chaque paire de fragments.
Tous les calculs de score peuvent se faire en parallèle.
Deux méthodes de parallélisation sont envisageables:
\begin{enumerate}
\item Pour chaque thread, on définit une part égale de l'ensemble des arcs à calculer.
\item À chaque fois qu'un thread a calculé un arc, il pioche une autre tâche à
	effectuer dans une mémoire partagée avec les autres threads.
	C'est exactement le problème des producteurs/consommateurs. %FIXME formulation ?
\end{enumerate}
Nous avons choisi la première méthode car nous observons que les différences
de temps de calcul entre chaque thread sont identiques, à 1 seconde près,
%TODO refaire les tests et indiquer les résultats
ce qui signifie que leur charge de travail est équilibrée.
De plus, beaucoup d'appels systèmes auraient eu lieu si la deuxième méthodes
avait été utilisée.

\subsection{Même score pour plusieurs arcs}
Les arcs utilisés sont orientés. Le poids d'un arc \arc{f}{g} est
le score de l'alignement où soit on aligne le suffixe de $f$ au préfixe de $g$,
soit $f$ est inclus à $g$. Calculons la matrice d'alignement à $|f|+1$ lignes et $|g|+1$ colonnes.
Le score de \arc{f}{g} est le maximum des scores sur la dernière ligne.
Il correspond aussi au score de \arc{\bar{g}}{\bar{f}} car l'alignement est exactement le même.\\

Le maximum des scores de la dernière colonne de la matrice donne celui de \arc{g}{f}.
En effet, si nous effectuons un alignement à partir d'un position sur la dernière colonne,
un suffixe de $g$ est entièrement aligné sur une partie de $f$.
Nous avons bien que soit le suffixe de $g$ est aligné avec le préfixe de $f$, soit $g$ est inclus à $f$.
Et puisque nous avons le score de \arc{g}{f}, nous avons aussi celui de \arc{\bar{f}}{\bar{g}}.
Quatre scores peuvent donc être calculés avec une seule matrice d'alignement.

\subsection{Typage des scores}
La valeur d'un score peut être représentée sur 2 octets (\texttt{short} en Java).
Effectivement, la taille maximale de fragment pour toutes les collections qui
nous ont été fournies est de 1366.
%TODO pas oublier de changer si changement des constantes de score
Une borne inférieure au score est -2732 (seulement des gaps)
et une borne supérieure est 1366 (alignement total).
Par conséquent, un \texttt{short} convient puisque sa valeur peut se trouver
entre -32768 et 32767.
